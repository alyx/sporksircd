\chapter{Extensions}
\label{extensions}

\section{Channel mode extensions}
\label{chmextensions}

	These extensions add various additional channel modes.

\subsection{chm\_adminonly, admins only (+A)}

	Adds +A, which blocks users who are not admins from joining a channel.

\subsection{chm\_blockcaps, restrict use of caps (+G)}

	Adds +G, which blocks all messages that contain too many uppercase
	letters.  The maximum percentage of caps allowed in a message is
	controlled by \nolinkurl{blockcaps::threshold}.
	\subsubsection{configuration}
	\begin{verbatim}
	blockcaps {
	        threshold = 75;
	};
	\end{verbatim}

\subsection{chm\_censor, censored words list (+y)}

	Adds +y, which allows channel operators to block specific words,
	phrases, or patterns from being used in a channel.  The configuration
	option \nolinkurl{censor::max\_censors} controls how many censors may
	be set in a channel, and \nolinkurl{censor::\_max\_censors\_large}
	controls the number of censors that may be set in a +L channel.
	\subsubsection{configuration}
		\begin{verbatim}
		censor {
		        max_censors = 25;
		        max_censors_large = 250;
		};
		\end{verbatim}

\subsection{chm\_censor\_operonly, restrict censor to opers (+W)}

	When this extension is loaded, channel operators may not use censor
	lists unless an oper sets +W on the channel.

	\notebox{Note}{This module is going away eventually.}

\subsection{chm\_nocolour, strip colour codes and other control codes (+c)}

	Adds +c, which strips coluor codes and control codes from messages
	sent to a channel.

\subsection{chm\_noctcp, no channel-wide CTCP (+C)}

	Adds +C, which blocks all CTCP messages sent to a channel (except
	for ACTION, used by the /me command).

\subsection{chm\_nonotices, no channel NOTICEs (+T)}

	Adds +T, which blocks all channel-wide NOTICEs.

\subsection{chm\_norepeat, no repeat messages (+K)}

	Adds +K, which filters any repeated messages.

\subsection{chm\_operonly, oper-only channel (+O)}

	Adds +O, which allows opers to block all users who are not opers
	from joining.

\subsection{chm\_operonly\_compat}

	Aliases +O to +b \$~o, which bans all non-opers.

	This requires \nolinkurl{extensions/extb\_opers.so}.

\subsection{chm\_quiet\_unreg\_compat}

	Aliases +R to +q \$~a, which mutes all unidentified users.

	This requires \nolinkurl{extensions/extb\_account.so}.

\subsection{chm\_sslonly, SSL-only channel (+S)}

	This mode blocks all users who are not using SSL.

\subsection{chm\_sslonly\_compat}

	Aliases +S to +b \$~z, which bans all users not using SSL.

	This requires \nolinkurl{extensions/extb\_ssl.so}.


\section{Usermode extensions}
\label{umodeextensions}

\subsection{usm\_botmode}

	Adds usermode +B, which allows a user to be identified as a bot.
	This confers no additional restrictions or privileges.

\subsection{usm\_ip\_cloaking}

	Adds usermode +x, which scrambles a user's real IP or hostname in WHOIS.
	Opers will still be able to see real IPs and hostnames on a separate line in
	WHOIS. To set +x on all users automatically, add +x to general::default\_umodes in
	ircd.conf

\subsection{usm\_override}

	Gives opers with the oper:override flag the ability to temporarily set usermode +p,
	which grants operator privileges on all channels as well as the ability to bypass
	bans, quiets, and other modes.  This expires after the time configured by the
	\nolinkurl{override::expire\_time} setting in ircd.conf, unless it is set
	to 0, in which case it will not expire. Additionally, if \nolinkurl{override::make\_immune}
	is true, then opers with +p cannot be kicked.
	\subsubsection{configuration}
		\begin{verbatim}
		override {
		        expire_time = 5 minutes;
		        make_immune = no;
		};
		\end{verbatim}


\section{Extended ban extensions}
\label{extbextensions}

	These extensions add support for various extended ban masks
	(extbans).

\subsection{extb\_account}

	Adds the \$a extban, which can be used in one of two ways.
	The first way, \$a, matches all identified users.  The second,
	\$a:<account> matches only users with a matching account name.
	Like all other extbans, this can be negated with a tilde (~),
	so that \$~a matches all unregistered and unidentified users.

\subsection{extb\_canjoin}

	Adds the \$j:<channel> extban, which matches users who can
	join a specifiedchannel.  It can be used to effectively
	synchronise the ban, quiet, exempt, and invex lists of a channel
	with those of another channel.  Like all other extbans, matches
	can be negated with a tilde (~), such that \$~j:<channel> matches
	all users who cannot join the specified channel.

\subsection{extb\_channel}

	Adds the \$c:<channel> extban, which matches users who are in
	a specified channel.  Matches can be negated with a tilde (~), so
	\$~c:<channel> matches all users who are not in the specified channel.

	If the channel specified is +s or +p, then this extban will not match
	regardless of whether or not the user being checked is in the specified
	channel.

\subsection{extb\_extgecos}

	Adds the \$x:<nick!user@host\#realname> extban, which extends the normal
	nick!user@host mask to include a realname.  In addition, the separators
	(!, @, and \#) are not mandatory as they are in normal masks.  This extban
	may be negated by placing a tilde (~) after the dollar sign (\$).

\subsection{extb\_oper}

	This adds the \$o extban, which matches all opers.  The opposite form,
	\$~o, matches all non-opers.

\subsection{extb\_realname}

	Adds the \$r:<realname> extban, which matches any user with a realname
	matching the specified realname pattern.  This extban can also be negated
	with a tilde (~) placed after the dollar sign (\$).

\subsection{extb\_server}

	Adds the \$s:<server> extban, which matches all users connected to a
	specified server.  This extban can also be negated.

\subsection{extb\_ssl}

	Adds the \$z extban, which matches all users who are connected using SSL.
	It can be used to make a channel SSL-only by banning \$~z.


\section{Command extensions}
\label{cmdextensions}

	These extensions add support for various commands.

\subsection{m\_adminwall}

	Adds /ADMINWALL , which works like operwall, but can only be sent and
	received by admins.

	Syntax: \command{ADMINWALL} \userliteral{message}

\subsection{m\_chgflags}

	Adds /CHGFLAGS, which allows an oper to grant various auth\{\} block flags
	to regular users.  Its use requires the oper:grant flag.

	Syntax: \command{CHGFLAGS} \userliteral{user} \userliteral{flags}

\subsection{m\_cycle}

	Adds /CYCLE, which allows a user to part then join a channel instantaneously.

	Syntax: \command{CYCLE} \userliteral{channel}

\subsection{m\_findforwards}

	Adds /FINDFORWARDS, which allows opers and chanops to find a channel's forwards.

	Syntax: \command{FINDFORWARDS} \userliteral{channel}

\subsection{m\_forcejoin}

	Allows opers with the oper:force flag to force a user to join a channel.

	Syntax: \command{FORCEJOIN} \userliteral{nick} \userliteral{channel}

\subsection{m\_forcenick}

	Allows opers with the oper:force flag to forcibly change a user's nick.

	Syntax: \command{FORCENICK} \userliteral{oldnick} \userliteral{newnick}

\subsection{m\_forcepart}

	Allows opers with the oper:force flag to force a user to part a channel.

	Syntax: \command{FORCEPART} \userliteral{nick} \userliteral{channel} [:\userliteral{reason}]

\subsection{m\_forcequit}

	Allows opers with the oper:force flag to force a user to quit.  Note that this may
	be abusive.

	\notebox{Note}{
		This module lives under unsupported.
	}

	Syntax: \command{FORCEQUIT} \userliteral{nick} [:\userliteral{reason}]

\subsection{m\_identify}

	Allows identifying to NickServ or ChanServ by using /IDENTIFY.

	Syntax: \command{IDENTIFY} \userliteral{nick|channel} \userliteral{password}

\subsection{m\_mkpasswd}

	Allows a user to generate a password hash that may be used in the ircd config.
	The supported hash types are SHA256, SHA512, BLOWFISH, and MD5.

	Syntax: \command{MKPASSWD} \userliteral{password} \userliteral{hashtype}

\subsection{m\_oaccept}

	Allows opers to forcibly add themselves to a user's ACCEPT list.

	Syntax: \command{OACCEPT} \userliteral{nick}

\subsection{m\_ojoin}

	Allows opers to join any channel regardless of any modes or bans set on the
	channel.

	Syntax: \command{OJOIN} [\userliteral{status}]\userliteral{channel}

\subsection{m\_olist}

	Allows opers to view all channels including hidden ones. Requires oper:spy.

	Syntax: \command{OLIST}

\subsection{m\_opme}

	Allows admins to op themselves on opless channels.

	Syntax: \command{OPME} \userliteral{channel}

\subsection{m\_omode}

	Allows admins to change modes on any channel.

	Syntax: \command{OMODE} \userliteral{channel} \userliteral{modestring}

\subsection{m\_roleplay}

	Adds many roleplay commands for use on channels with +N set.

	\subsubsection{roleplay commands}
		\command{SCENE} \userliteral{channel} \userliteral{scene-description}
		\command{FSAY} \userliteral{channel} \userliteral{nick} :\userliteral{message}
		\command{FACTION} \userliteral{channel} \userliteral{nick} :\userliteral{action}
		\command{NPC} \userliteral{channel} \userliteral{npc-name} :\userliteral{message}
		\command{NPCA} \userliteral{channel} \userliteral{npc-name} :\userliteral{action}

	Note that FSAY and FACTION are only allowed if m\_roleplay.c is compiled with the
	ABUSIVE\_ROLEPLAY macro defined.

\subsection{m\_remove}

	Allows channel operators to use the REMOVE command, which forces a user to part a channel.
	Unlike a KICK, a REMOVE is seen by other users in the channel and the target as a regular PART.
	Because it uses PART in a nonstandard way, it may cause some clients to malfunction, but this
	does have the effect of defeating most clients' autorejoin capabilities.

	A REMOVE will be translated as a KICK for servers on the network that do not have \nolinkurl{extensions/m\_remove.so}
	loaded.

	Syntax: \command{REMOVE} \userliteral{channel} \userliteral{nick} [:\userliteral{reason}]

\subsection{m\_sendbans}

	Allows opers with the ability to remotely set bans, resvs, and xlines to send
	resvs and xlines to remote servers.  This module probably is not needed.

	Syntax: \command{SENDBANS} \userliteral{server-mask}

\subsection{m\_webirc}

	Adds support for WEBIRC, which allows a web-based gateway to send a user's real IP or
	hostname to the server.  To use it, copy the following config block into your ircd.conf
	and edit it to meet your needs:

\begin{verbatim}
auth {
	/* user: webirc@IP.OF.YOUR.WEBIRC . the webirc@ part is required */
	user = "webirc@192.168.1.1";
	password = "<password>";
	/* The spoof must be set to "webirc." in order to work */
	spoof = "webirc.";
	class = "users";
};
\end{verbatim}


\section{Server notice mask extensions}
\label{snoextensions}

	These extensions add support for various server notice masks (snomasks).

\subsection{sno\_farconnect, +F}

	Allows opers to use snomask +F, which shows remote connections and quits.

\subsection{sno\_globalkline}

	Notifies opers of global K/D/X-lines.

\subsection{sno\_globaloper}

	Notifies opers when remote opers sign in.

\subsection{sno\_whois, +W}

	Notifies opers who are using snomask +W when someone uses WHOIS on them.

\subsection{sno\_channeljoin, +j}

	Notifies opers who are using snomask +j when someone joins or parts a channel.


\section{Spy extensions}

	These extensions notify opers with the oper:spy flag and the +y snomask
	when certain commands are run.

\subsection{spy\_admin\_notice}

	Notifies opers when ADMIN is used.

\subsection{spy\_info\_notice}

	Notifies opers when INFO is used.

\subsection{spy\_links\_notice}

	Notifies opers when LINKS is used.

\subsection{spy\_motd\_notice}

	Notifies opers when MOTD is used.

\subsection{spy\_stats\_notice}

	Notifies opers when STATS is used.

\subsection{spy\_stats\_p\_notice}

	Notifies opers when STATS p is used, which lists online opers.
	It will not notify opers when any other STATS are requested.
	Note that this will generate two notices for STATS p if
	\nolinkurl{extensions/spy\_stats\_notice.so} is also loaded.

\subsection{spy\_trace\_notice}

	Notifies opers when TRACE is used.


\section{Other extensions}
\label{miscextensions}

	These extensions do not fit into any of the above categories.

\subsection{createauthonly}

	Only allows users who have identified with services to create channels.

\subsection{createoperonly}

	Only allows opers to create channels.

\subsection{example\_module}

	An example module.  This does nothing useful.

\subsection{force\_user\_invis}

	Forces users to always have usermode +i set.

\subsection{hurt}

	Adds the HURT system, which allows opers to require that users matching
	certain masks identify immediately or be disconnected.  Setting a HURT
	requires oper:kline, while removing it via HEAL requires oper:unkline.

	HURT Syntax: \command{HURT} \userliteral{ip} \userliteral{expire-time} :\userliteral{reason}

	HEAL Syntax: \command{HEAL} \userliteral{(ip|nick)}

	\notebox{Note}{
		Hurts are not bursted and are not saved to bandb. This will be
		fixed in a later version.
	}

\subsection{no\_locops}

	Forbids opers from using usermode +l, which allows sending and receiving LOCOPS.

\subsection{no\_oper\_invis}

	Forbids opers from using usermode +i.

